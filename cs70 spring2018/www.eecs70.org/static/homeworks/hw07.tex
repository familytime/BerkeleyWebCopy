\Question{Counting, Counting, and More Counting}

The only way to learn counting is to practice, practice, practice, so
here is your chance to do so.
For this problem, you do not need to show work that justifies your answers.
We encourage you to leave your answer as an expression (rather than
trying to evaluate it to get a specific number).
\begin{Parts}
%\Part How many 10-bit strings are there that contain exactly 4 ones?

\Part How many ways are there to arrange $n$ 1s and $k$ 0s into a sequence?

\Part A bridge hand is obtained by selecting 13 cards from a standard
  52-card deck. The order of the cards in a bridge hand is
  irrelevant. \\
  How many different 13-card bridge hands are there? 
  How many different 13-card bridge hands are there that contain
  no aces? How many different 13-card bridge hands are there that contain
  all four aces? How many different 13-card bridge hands are there that contain
  exactly 6 spades?

\Part Two identical decks of 52 cards are mixed together, yielding a
  stack of 104 cards.
  How many different ways are there to order this stack of 104 cards?
  
\Part How many 99-bit strings are there that contain more ones than
  zeros?
  
\Part An anagram of FLORIDA is any re-ordering of the letters of FLORIDA, i.e., any
  string made up of the letters F, L, O, R, I, D, and A, in any order.
  The anagram does not have to be an English word. \\
  How many different anagrams of FLORIDA are there? How many different anagrams 
  of ALASKA are there? How many different anagrams of ALABAMA are there? 
  How many different anagrams of MONTANA are there?
%\Part If we have a standard 52-card deck, how many ways are there to
%  order these 52 cards?
 
\Part How many  different anagrams of ABCDEF are there if: (1) C is the left neighbor of E; (2) C is on the left of E.

\Part We have 9 balls, numbered 1 through 9, and 27 bins.
  How many different ways are there to distribute these 9 balls among
  the 27 bins? Assume the bins are distinguishable (e.g., numbered 1
  through 27).
  
\Part We throw 9 identical balls into 7 bins.
  How many different ways are there to distribute these 9 balls among
  the 7 bins such that no bin is empty? Assume the bins are
  distinguishable (e.g., numbered 1 through 7).
  
\Part How many different ways are there to throw 9 identical balls
  into 27 bins? Assume the bins are distinguishable (e.g., numbered 1
  through 27).
  
\Part There are exactly 20 students currently enrolled in a class.
  How many different ways are there to pair up the 20 students, so
  that each student is paired with one other student?

% \Part Let (1, 1) be the bottom-left corner and (8, 8) be the upper-right 
% corner of a chessboard. If you are allowed to move one square at a time and
% can only move up or right, what is the number of ways to go from the bottom-left corner to 
% the upper-right corner? 

% \Part What is the number of ways to go from the bottom-left corner to 
% the upper-right corner of the chesssboard, if you must pass through the square 
% (6, 2), where $(i, j)$ represents the square in the $i$th row from the
% bottom and the $j$th column from the left? (Again, you are only allowed to move up or right.)
  
\Part How many solutions does $x_0 + x_1 + \cdots + x_k = n$ have, if each $x$ must be a non-negative integer?

\Part How many solutions does $x_0 + x_1 = n$ have, if each $x$ must be a \emph{strictly positive} integer?

\Part How many solutions does $x_0 + x_1 + \cdots + x_k = n$ have, if each $x$ must be a \emph{strictly positive} integer?

\end{Parts}


\Question{N-Bit} Let $NBit$ be a program that takes in a program $P$ and an input $x$ and outputs ``true'' if the program ever directly assigns a variable to an $n$-bit number and ``false'' otherwise. Can $NBit$ exist? If so, describe the procedure, and if not, prove that it cannot exist. (\textit{Hint: model off of the proof that $TestHalt$ cannot exist})

\Question{Counting Cartesian Products}

For two sets $A$ and $B$, define the Cartesian Product as $A \times B = \{(a,b) : a \in A, b \in B \}$.

\begin{Parts}
	\Part Given two countable sets $A$ and $B$, prove that $A \times B$ is countable.
	\Part Given a finite number of countable sets $A_1, A_2, \dots, A_n$, prove that 
	$A_1 \times A_2 \times \cdots \times A_n$ \\is countable. 
	\Part Consider an infinite number of countable sets: $B_1, B_2, \dots$. Under what
	condition(s) is \\$B_1 \times B_2 \times \cdots$ countable? Prove that if this
	condition is violated, $B_1 \times B_2 \times \cdots$ is uncountable.
\end{Parts}

\Question{Vacation Time}
After a number of complaints, the Dunder Mifflin Paper Company has decided on the following rule for vacation leave for the next year (365 days): Every employee must take exactly one vacation leave of 4 consecutive days, one vacation leave of 5 consecutive days and one vacation leave of 6 consecutive days within the year, with the property that any two of the vacation leaves have a gap of at least 7 days between them. In how many ways can an employee arrange their vacation time? (The vacation policy resets every year, so there is no need to worry about leaving a gap between this year and next year's vacations).

\Question{Fermat's Wristband}

Let $p$ be a prime number and let $k$ be a positive integer.
We have beads of
$k$ different colors, where any two beads of the same color are indistinguishable.

\begin{Parts}
    \Part
    We place $p$ beads onto a string.
    How many different ways are there construct such a sequence of $p$ beads with up to $k$ different colors?

    \Part 
    How many sequences of $p$ beads on the string are there that use at least two colors?

    \Part
    Now we tie the two ends of the string together, forming a
    wristband.
    Two wristbands are equivalent if the sequence of colors on one
    can be obtained by rotating the beads on the other.
    (For instance, if we have $k=3$ colors, red (R), green (G), and
    blue (B), then the length $p = 5$ necklaces RGGBG, GGBGR, GBGRG, BGRGG, and GRGGB are all
    equivalent, because these are all rotated versions of each other.)

    How many non-equivalent wristbands are there now?
    Again, the $p$
    beads must not all have the same color.
    (Your answer should be a simple function of $k$ and $p$.)

    [\textit{Hint}: Think about the fact that rotating all the beads on the wristband to another
        position produces an identical wristband.]

    \Part Use your answer to part (c) to prove Fermat's little theorem.
\end{Parts}


\Question{Story Problems}

\newcommand{\sblank}{\vspace{1in}}
Prove the following identities by combinatorial argument:
\begin{Parts}

    %\Part $\binom{n}{k} = \binom{n-1}{k-1} + \binom{n-1}{k}$

  	\Part $\binom{2n}{2} = 2 \binom{n}{2} + n^2$

  	\Part $n^2 = 2 \binom{n}{2} + n$

    \Part $\sum_{k=0}^n k {n \choose k} = n2^{n-1}$ \\
    \textit{Hint:} Consider how many ways there are to pick groups of people ("teams") and then a representative ("team leaders").

    \Part $\sum_{k=j}^n {n \choose k} {k \choose j} = 2^{n-j} {n \choose j}$ \\
    \textit{Hint:} Consider a generalization of the previous part.

\end{Parts}

% Carries over from 10M


